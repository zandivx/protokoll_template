\documentclass[11pt]{scrartcl}

% standard packages
\usepackage[utf8]{inputenc}  % input in UTF-8
\usepackage[T1]{fontenc}  % output in T1 fonts (westeuropäische Codierung)
\usepackage{lmodern}  % latin modern fonts
\usepackage[ngerman]{babel}  % deutsches Sprachpaket, neue Rechtschreibung

% Seitensetup
\usepackage{scrlayer-scrpage}  % Seitenformatierung durch KOMA-interne Optionen
\usepackage[top=4cm, bottom=4cm]{geometry}  % Seitengeometrie (kann durch KOMA ersetzt werden, hab ich aber nicht geschafft)
\usepackage[hypcap=false, labelfont=bf]{caption, subcaption}  % caption editing - hypcap warning with hyperref - captions fett
\usepackage{array}  % table editing

% additional packages
\usepackage{amsmath, amssymb, amstext}  % math packages (American Math Society)
\usepackage{mathtools}  % verbessert package amsmath
\usepackage{icomma}  % Kommata in Dezimalzahlen verursachen keinen Abstand mehr
\usepackage{graphicx}  % Bilder einfügen
\usepackage{pdfpages}  % PDF als vollständige Seiten einfügen
\usepackage{lastpage}  % referenziert die letzte Seite
\usepackage{siunitx}  % bessere Darstellung von Einheiten
\usepackage[hidelinks]{hyperref}  % should be last to loaded, produces most errors

% Kopf- und Fußzeile durch KOMA
\pagestyle{scrheadings}  % KOMA darf entscheiden
\clearpairofpagestyles  % reset
\setkomafont{pageheadfoot}{\normalfont}  % Standardschrift in Kopf- und Fußzeile
\setlength{\headheight}{27.2pt}  % benötigte Höhe Kopfzeile (warning von scrlayer-scrpage, wird aber automatisch so gerendert, falls diese Option weggelassen wird)
\ihead{Versuch: \\ Versuchsname}  % Kopf links
\chead{\textsc{Nachname1} Vorname1 \\ \textsc{Nachname2} Vorname2}  % Kopf Mitte
\ohead{Datum: \\ Datum der Messung}  % Kopf rechts
\cfoot{\pagemark \, / \pageref{LastPage}}  % Fuß Mitte

% Table of Contents
\DeclareTOCStyleEntry{dottedtocline}{section}  % KOMA intern - Inhaltsverzeichnis mit Punkten (nur sections)

% SI
\sisetup{locale = DE}

% array
\renewcommand{\arraystretch}{1.2}  % stretched die Zellengröße in tables auf das 1,2-fache

% neue Namen (babel überschreibt Standardbezeichnungen aus dem Englischen am Dokumentanfang - therefore gehören die neuen commands erst am Dokumentbeginn ausgeführt)
%\AtBeginDocument{\renewcommand{\contentsname}{Inhaltsangabe}}
%\AtBeginDocument{{\renewcommand{\listtablename}{Tabellenverzeichnis}}
%\AtBeginDocument{{\renewcommand{\listfigurename}{Abbildungsverzeichnis}}

% eigene Commands
%\newcommand{\der}[2]{\frac{\mathrm{d}#1}{\mathrm{d}#2}}  % derivative
%\newcommand{\pder}[2]{\frac{\partial #1}{\partial #2}}  % partial derivative



\begin{document}

\includepdf{deckblatt.pdf}

\tableofcontents
\newpage

\section{Aufgabenstellung}
\label{sec:aufgabenstellung}

Text


\section{Voraussetzungen und Grundlagen}
\label{sec:voraussetzungen-rundlagen}

Text


\section{Versuchsanordnung}
\label{sec:versuchsanordnung}

% Gleichungsumgebung
\begin{equation}
\label{eq:grenzen-oben-unten}
    \int \limits_{a}^{b} x^2 \, dx =\frac{b^3-a^3}{3}  % \, liefert quasi ein Leerzeichen - single character exponent muss nicht in {}-Klammern
\end{equation}

% inline math
\( \lim \limits_{n \to \infty} \left( 1 + \frac{1}{n} \right) ^{n} = e\)  \\ % new, encouraged
$\lim \limits_{n \to \infty} \frac{1}{2n} = 0$  % old, disencouraged

% display math
\begin{displaymath}
    \frac{df}{dx} = f'
\end{displaymath}

\[ \int \frac{1}{x} \, dx = \ln|x| \]  % new, encouraged - same as \begin{equation*} when using amsmath
$$ \textbf{F} = m \cdot \textbf{a} $$  % old, very disencouraged



\section{Geräteliste}
\label{sec:geraeteliste}

% Beispieltabelle 
\begin{center}
\captionof{table}[Geräteliste]{Verwendete Geräte + genaue, lange Beschreibung}  % optionales Argument wird in Verzeichnissen verwendet, verpflichtendes Argument direkt im Text
\label{tab:geraeteliste}
    \begin{tabular}{|c|c|c|c|} \hline
        Gerät & Hersteller & Modell & Genauigkeitsklasse \\ \hline
        Der Gerät & Schweißfrei Inc. & Nie-Müde & 12 \\ \hline
    \end{tabular}
\end{center}

\section{Versuchsdurchführung und Messergebnisse}
\label{sec:versuchsdurchfuehrung-messergebnisse}

Hello Test Miau

\section{Auswertung}
\label{sec:auswertung}

\section{Diskussion und Zusammenfassung}
\label{sec:diskussion-zusammenfassung}

% Literaturtabelle
%\bibliographystyle{unsrt}
%\bibliography{Literatur}
\listoffigures
\listoftables

\end{document}
