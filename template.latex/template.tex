% Protokoll-Template
% by Andreas Zach


% class
\documentclass[ngerman]{scrartcl}


% input and font
\usepackage[utf8]{inputenc}  % input in UTF-8
\usepackage[T1]{fontenc}  % output in T1 fonts (west European encoding)
\usepackage{lmodern}  % Latin modern font
\usepackage[mono]{zi4}  % Consolas font for monospaced text
\usepackage{babel}  % language package
\usepackage{amsmath, amssymb, amstext, mathtools}  % math packages (American Math Society) + correction of amsmath (mathtools) [loads amsmath automatically]
\usepackage{letltxmacro}  % \let command for robust macros (new sqrt)


% page geometry
\usepackage{scrlayer-scrpage}  % page formatting with KOMA options
\usepackage{geometry}


% floats
\usepackage[hypcap=false, labelfont=bf]{caption, subcaption}  % caption editing - hypcap warning with hyperref
\usepackage{float}  % for [H] (forced here) specifier
\usepackage{tabularray}
\usepackage{diagbox}  % table cells with diagonal lines


% input
\usepackage{graphicx}  % input JPEG, PNG, PDF, etc.
\usepackage{pdfpages}  % input PDF as whole pages
\usepackage{lastpage}  % reference to last page


% text
\usepackage[locale=DE, uncertainty-mode=separate]{siunitx}  % SI units, German formatting - \pm stays \pm instead of ..(.)
\usepackage{icomma}  % no space after commas instead of English points) in decimal values
\usepackage{enumitem}  % better enumerating with style options
\usepackage{nicefrac}  % inline-fractions in n/d-style
\usepackage{xcolor}  % custom colors
\usepackage{listings, scrhack}  % code display; listings in combination with KOMA
\usepackage{fancyvrb}  % Verbatim environment with better options (capital V!)


% literacy
\usepackage[style=apa]{biblatex}  % backend=Biber is standard
\usepackage{csquotes}  % better quotation - should also be used in combination with package babel (warning)
\usepackage{xurl}  % breaks links - after BibLaTeX, but before hyperref!
\usepackage[hidelinks]{hyperref}  % produces most errors, last to load


% KOMA setups
% header and footer
\pagestyle{scrheadings}  % KOMA style
\clearpairofpagestyles  % reset
\setkomafont{pageheadfoot}{\normalfont}  % standard font in header and footer
\setlength{\headheight}{27.2pt}  % just look at the warning
\ihead{Versuchsname}  % inner (left) head
\chead{\textsc{Zach} Andreas \\ 12004790}  % center head
\ohead{Datum der Messung}  % outer (right) head
\cfoot{\pagemark{} / \pageref*{LastPage}}  % center foot - *: ref but no hyperlink
% {}: empty statement
% \ : protected space
% \,: small space
\DeclareTOCStyleEntry{dottedtocline}{section}  % sections in TableOfContents with dotted lines
\KOMAoptions{parskip=half-}  % paragraphs with half a line height space instead of indentation, last line with no special treatment


% package setups

% BibLaTeX source
\addbibresource{C:/Users/andre/Dropbox/Uni/Physik/Allgemeines/protokoll_template/files/JabRef_Database/physics.bib}  % database import with absolute path - file ending!


% rewrite names (babel overwrites German with standard English names, therefore at document beginn [after everything is loaded])
\AtBeginDocument{\renewcommand{\refname}{Literaturverzeichnis}}
% others:
% \contentsname
% \listtablename
% \listfigurename


% xcolor
\definecolor{code_keyword}{HTML}{A06E9D}
\definecolor{code_string}{HTML}{AD6E3E}
\definecolor{code_comment}{HTML}{6A9955}
%\definecolor{code_basic}{HTML}{D4D4D4}
%\definecolor{code_background}{HTML}{1E1E1E}


% listings
\lstdefinestyle{python}{
    basicstyle=\fontfamily{zi4}\footnotesize,%\color{code_basic},  % \footnotesize contains \selectfont implicitly
    %backgroundcolor=\color{code_background},
    commentstyle=\color{code_comment},
    keywordstyle=\bfseries\color{code_keyword},
    numberstyle=\tiny,
    stringstyle=\color{code_string},
    breakatwhitespace=false,
    breaklines=true,
    captionpos=b,
    keepspaces=true,
    numbers=left,
    numbersep=5pt,
    showspaces=false,
    showstringspaces=false,
    showtabs=false,
    tabsize=2,
}
\lstset{style=python}
\renewcommand*{\ttdefault}{lmvtt}  % Latin Modern Typewriter Proportional


% new sqrt
% https://en.wikibooks.org/wiki/LaTeX/Mathematics
\makeatletter
\let\oldr@@t\r@@t
\def\r@@t#1#2{%
\setbox0=\hbox{$\oldr@@t#1{#2\,}$}\dimen0=\ht0
\advance\dimen0-0.2\ht0
\setbox2=\hbox{\vrule height\ht0 depth -\dimen0}%
{\box0\lower0.4pt\box2}}
\LetLtxMacro{\oldsqrt}{\sqrt}
\renewcommand*{\sqrt}[2][\ ]{\oldsqrt[#1]{#2} }
\makeatother

 
% tabularray
% imports_and_setups{
%     expl3,
%     xparse,
%     ninecolors
%     \hypersetup{pdfborder={0 0 0}}
% }
\RequirePackage{tabularray}  % like \usepackage
\UseTblrLibrary{diagbox}  % diagbox suited for tabularray
\UseTblrLibrary{siunitx}  % siunitx suited for tabularray
% TRIPLE CURLY BRACKETS {{{}}} to protect strings from \num{} interpretation in S columns

% standard environment
\SetTblrInner{
    hlines,
    vlines,
    columns={
            halign=c,
            valign=m,
        },
}

% X columns
\NewTblrEnviron{tblrx}
\SetTblrInner[tblrx]{
    hlines,
    vlines,
    columns={
            halign=c,
            valign=m,
            co=1,  % coefficient of width for expendable columns (X columns)
        },
    width=\linewidth,
    vspan=minimal,
}

% -X columns
\NewTblrEnviron{tblr-x}
\SetTblrInner[tblr-x]{
    hlines,
    vlines,
    columns={
            halign=c,
            valign=m,
            co=-1,  % shrinks X column down to natural width
        },
    width=\linewidth,
    vspan=minimal,
}

% no hline and vline left and on top of first cell
\NewTblrEnviron{tblr_omit_first_cell}
\SetTblrInner[tblr_omit_first_cell]{
    hlines,
    vlines,
    columns={
            halign=c,
            valign=m,
        },
    hspan=even,
    vspan=minimal,
    %
    hline{1}={1}{white},  % first row, only first cell
    vline{1}={1}{white},
}


% own commands
\newcommand*{\der}[3][]{\frac{\mathrm{d}^{#1}#2}{\mathrm{d}#3^{#1}}}  % derivative
\newcommand*{\pder}[3][]{\frac{\partial^{#1}#2}{\partial#3^{#1}}}     % partial derivative
% \newcommand* can't contain multiple lines
% \newcommand can





\begin{document}

\includepdf{deckblatt.pdf}

\clearpage
\tableofcontents
\newpage

\section{Aufgabenstellung}
\label{sec:aufgabenstellung}



\section{Grundlagen und Voraussetzungen}
\label{sec:grundlagen_voraussetzungen}

Text1\footcite[1000]{ref:dem1} \\
Text2\footcite[Kapitel 74]{ref:knoll} \\
Text3\footnote{\url{https://online.uni-graz.at/kfu_online/ee/ui/ca2/app/desktop/\#/login?$ctx=&redirect=Li4vLi4vLi4vZWUvdWkvY2EyL2FwcC9kZXNrdG9wLyMvc2xjLnRtLmNwL3N0dWRlbnQvY291cnNlcy82Mjg3OTk=}} \\  % must escape the # (\#), error otherwise!!
Text4\footcite{ref:genol2013}
%
% Gleichungsumgebung
\begin{equation}
    \label{eq:grenzen-oben-unten}
    \int \limits_{a}^{b} x^2 \, \mathrm{d}x =\frac{b^3-a^3}{3}  % \, -> small space, single character exponent doesn't need curly braces
\end{equation}
%
% inline math
Inline math: \(\lim \limits_{n \to \infty} \left( 1 + \frac{1}{n} \right) ^{n} = e\)  \\ \\ % new, encouraged
Inline math: $\lim \limits_{n \to \infty} \frac{1}{2n} = 0$  % old, discouraged
%
% display math
\begin{displaymath}
    \sqrt[3]{27} = 3 \implies A \times B \implies \SI{30.0(2)}{\meter\per\second}
\end{displaymath}
%
\[ \int \frac{1}{x} \, dx = \ln|x| \]  % new, encouraged - same as \begin{equation*} when using amsmath
$$ \textbf{F} = m \cdot \textbf{a} $$  % old, very discouraged



\section{Versuchsanordnung}
\label{sec:versuchsanordnung}

\begin{figure}[H]
    \centering
    \begin{samepage}
        \includegraphics[width=0.6\linewidth]{example-image-golden}
        \caption{Example image golden}
        \label{fig:example-image-golden}
    \end{samepage}
\end{figure}



\section{Geräteliste}
\label{sec:geraeteliste}

\begin{table}[H]
    \centering
    \begin{samepage}  % caption and table on same page
        \caption[Geräteliste]{Verwendete Geräte und wichtige Materialien}  % optional argument for List of Tables, mandatory argument for caption
        \label{tab:geraeteliste}
        \begin{tblrx}{
                colspec={
                        *{3}{Q}  % multiplication syntax: *{n}{thing}
                        S[table-format=2.2]
                        Q[co=0]  % Q column for not overwriting standard values of preamble (no extendable width in order to work with diagbox)
                    },
                cell{2}{2}={r=2}{},  % empty specifier for not overwriting standard values of preamble
                cell{4}{3}={r=2,c=2}{},
            }
            Gerät   & Hersteller & Modell   & {{{Unsicherheit}}}    & Anmerkung                             \\
            Gerät 1 & ich        & meins    & 0.01                  & {quasi \\ perfekt \\ genau}           \\
            Gerät 2 &            & passt so & 21.4                  & quasi perfekt genau                   \\
            Gerät 3 & -          & passt so &                       & $\nabla$                              \\
            Gerät 4 & -          &          &                       & \diagbox[dir=NE]{Alle meine}{Entchen} \\  % if diagbox is not the widest column, width must be typeset manually -.-
        \end{tblrx}
    \end{samepage}
\end{table}



\section{Versuchsdurchführung und Messergebnisse}
\label{sec:versuchsdurchfuehrung_messergebnisse}



\section{Auswertung}
\label{sec:auswertung}



\section{Zusammenfassung und Diskussion}
\label{sec:zusammenfassung_diskussion}



\addsec{Python-Skript}
\label{sec:python}

\lstinputlisting[language=Python]{../labtool.python/labtool/__init__.py}



\clearpage
% Literaturverzeichnis
\printbibliography

% Abbildungsverzeichnis
\listoffigures

% Tabellenverzeichnis
\listoftables

\end{document}
