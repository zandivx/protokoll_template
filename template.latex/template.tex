% Protokoll-Template
% by Andreas Zach


% class
\documentclass[ngerman]{scrartcl}

% input preamble
\usepackage{iftex}

% text input and font
\ifluatex  % LuaLaTeX
    \usepackage{fontspec}
    % main font automatically: Latin Modern
    \IfFontExistsTF{Fira Code}{% true branch
        \setmonofont{Fira Code}[
            Contextuals=Alternate,  % Activate the calt feature
            StylisticSet={1,3,5,8}, % fontspec docs S. 46
            CharacterVariant={16}, % fontspec docs S. 37
            Numbers={SlashedZero} % fontspec docs S. 44
        ]}{% false branch
    }
\else  % pdfLaTeX
    \usepackage[utf8]{inputenc}  % input in UTF-8
    \usepackage[T1]{fontenc}  % output in T1 fonts (west European encoding)
    \usepackage{lmodern}  % Latin modern font for main text
    \IfFileExists{fira.sty}{% true branch
        \usepackage[mono]{fira}  % Fira (not Code!) font for monospaced text
    }{% false branch
    }
\fi

% text processing
\usepackage{babel}  % language package
\usepackage[intlimits]{mathtools}  % upgrade of amsmath (automatically loaded) - \int^_ like \limits^_
\usepackage{amssymb}  % upgrade of amsfonts (American Math Society)
\usepackage{amstext}  % \text command in math environments
\usepackage{letltxmacro}  % \let command for robust macros (new sqrt)
\usepackage{chemformula}  % typeset chemical formulas


% page geometry
\usepackage{scrlayer-scrpage}  % page formatting with KOMA options
\usepackage[paper=a4paper, hmargin=3cm, vmargin=2.5cm, includehead, includefoot]{geometry}  % horizontal: 3cm, vertical: 2.5cm strict with or without headers and footers
\usepackage{tabto}  % tab stops
\NumTabs{8}  % 8 equally spaced of \textwidth tab stops



% floats
\usepackage[hypcap=false, labelfont=bf]{caption, subcaption}  % caption editing - hypcap warning with hyperref
% counter prefixed with section number and therefore reset at each section:
\counterwithin{figure}{section}
\counterwithin{table}{section}
\usepackage{float}  % for [H] (forced here) specifier
\usepackage{tabularray}  % better tables


% graphical input
\usepackage{graphicx}  % input JPEG, PNG, PDF, etc.
\usepackage{pdfpages}  % input PDF as whole pages
\usepackage{lastpage}  % reference to last page
\usepackage{import} % include files from other directories


% text
\usepackage[locale=DE, uncertainty-mode=separate]{siunitx}  % SI units, German formatting - \pm stays \pm instead of ..(.)
\let\sqty\qty  % physics overrides \qty of siunitx, therefore make it available as \sqty
\usepackage{physics}  % macros for easier typesetting of physical formulas
\usepackage{icomma}  % no space after commas instead of English points) in decimal values
\usepackage{enumitem}  % better enumerating with style options
\usepackage{nicefrac}  % inline-fractions in n/d-style
\usepackage{xcolor}  % custom colors
\usepackage{listings, scrhack}  % code display; listings in combination with KOMA
\ifluatex
    \IfFontExistsTF{Fira Code}{%
        \usepackage[verbatim]{lstfiracode}  % Fira Code in listings
        \lstset{style=FiraCodeStyle}
    }{}
\fi
\usepackage{fancyvrb}  % Verbatim environment with better options (capital V!)


% literacy
\usepackage[sorting=none, giveninits=true]{biblatex}  % defaults: backend=Biber, style=numeric
% bibliography styles: https://www.overleaf.com/learn/latex/Biblatex_bibliography_styles
% citation styles: https://www.overleaf.com/learn/latex/Biblatex_citation_styles
\usepackage{csquotes}  % better quotation - should also be used in combination with package babel (warning)
\usepackage{xurl}  % breaks links - after BibLaTeX, but before hyperref!
\usepackage[hidelinks]{hyperref}  % produces most errors, last to load


% enumerate paragraphs and subparagraphs
% depths: https://www.overleaf.com/learn/latex/Sections_and_chapters
% -1 \part{part}
%  0 \chapter{chapter}
%  1 \section{section}
%  2 \subsection{subsection}
%  3 \subsubsection{subsubsection}
%  4 \paragraph{paragraph}
%  5 \subparagraph{subparagraph}
\setcounter{secnumdepth}{3}


% KOMA setups
% header and footer
\pagestyle{scrheadings}  % KOMA style
\clearpairofpagestyles  % reset
\setkomafont{pageheadfoot}{\normalfont}  % standard font in header and footer
\setlength{\headheight}{27.2pt}  % warning
\cfoot{\pagemark{} / \pageref*{LastPage}}  % center foot - *: ref but no hyperlink
% {}: empty statement
% \ : protected space
% \,: small space
\DeclareTOCStyleEntry[linefill=\TOCLineLeaderFill]{tocline}{section}  % sections in TableOfContents with dotted lines
% source: https://tex.stackexchange.com/a/651532
\KOMAoptions{parskip=half-}  % paragraphs with half a line height space instead of indentation, last line with no special treatment


% package setups

% rewrite names (babel overwrites German with standard English names, therefore at document beginn [after everything is loaded])
\AtBeginDocument{\renewcommand{\refname}{Literaturverzeichnis}}
% others:
% \contentsname
% \listtablename
% \listfigurename

% make title in bibliography upright
\DeclareFieldFormat{title}{#1}  % https://tex.stackexchange.com/a/311837
% make size of url in bibliography smaller
\renewcommand{\UrlFont}{\footnotesize\ttfamily}  % https://tex.stackexchange.com/a/151115, https://www.overleaf.com/learn/latex/Font_sizes%2C_families%2C_and_styles


% xcolor
\definecolor{code_keyword}{HTML}{A06E9D}
\definecolor{code_string}{HTML}{AD6E3E}
\definecolor{code_comment}{HTML}{6A9955}
% \definecolor{keyword_pink}{HTML}{c678dd}
% \definecolor{vscode_bg}{HTML}{282c34}
% \definecolor{vscode_var}{HTML}{e06c75}
% \definecolor{vscode_comment}{HTML}{7f848e}
% \definecolor{vscode_constant}{HTML}{d19a66}
% \definecolor{vscode_function}{HTML}{61afe3}
% \definecolor{background_grey}{HTML}{f8f8f8}
% \definecolor{code_basic}{HTML}{D4D4D4}
% \definecolor{code_background}{HTML}{1E1E1E}

% custom siunitx units
\DeclareSIUnit{\dig}{dig}  % digits for uncertainty of electronical measurement devices
\DeclareSIUnit{\px}{px}  % pixels
\sisetup{table-align-uncertainty=true}


% listings
\lstset{
    basicstyle=\ttfamily\footnotesize,%\color{code_basic},  % \footnotesize contains \selectfont implicitly
    %backgroundcolor=\color{code_background},
    commentstyle=\color{code_comment},
    keywordstyle=\bfseries\color{code_keyword},
    numberstyle=\tiny,
    stringstyle=\color{code_string},
    breakatwhitespace=false,
    breaklines=true,
    captionpos=b,
    keepspaces=true,
    numbers=left,
    numbersep=5pt,
    showspaces=false,
    showstringspaces=false,
    showtabs=false,
    tabsize=2
}


% new sqrt
% https://en.wikibooks.org/wiki/LaTeX/Mathematics
\makeatletter
\let\oldr@@t\r@@t
\def\r@@t#1#2{%
    \setbox0=\hbox{$\oldr@@t#1{#2\,}$}\dimen0=\ht0
    \advance\dimen0-0.2\ht0
    \setbox2=\hbox{\vrule height\ht0 depth -\dimen0}%
    {\box0\lower0.4pt\box2}}
\LetLtxMacro{\oldsqrt}{\sqrt}
\renewcommand{\sqrt}[2][\ ]{\oldsqrt[#1]{#2} }
\makeatother


% own commands
% \newcommand* can't contain multiple lines
% \newcommand can
\newcommand*{\mup}[1]{\ensuremath{\text{\textup{#1}}}}  % math mode upright normal font
\newcommand*{\inkgraphics}[3][\linewidth]{\def\svgwidth{#1}\import{#2}{#3}}


% custom tabularray environments

% imports and setups of tabularray: {
%     expl3,
%     xparse,
%     ninecolors
%     \hypersetup{pdfborder={0 0 0}}
% }

% additionally loaded libraries:
% diagbox
% varwidth
% booktabs
% counter

\UseTblrLibrary{amsmath}  % +array, +matrix, +bmatrix, +Bmatrix, +pmatrix, +vmatrix, +Vmatrix and +cases like tabularray with graphical options
\UseTblrLibrary{siunitx}  % siunitx suited for tabularray
\UseTblrLibrary{diagbox}  % table cells with diagonal lines, suited for tabularray
\UseTblrLibrary{varwidth}  % measure cell width
\UseTblrLibrary{booktabs}
\UseTblrLibrary{counter}


% custom tabularray environments:

% info:
% colcycle: https://github.com/lvjr/tabularray/issues/74
% guard: https://github.com/lvjr/tabularray/issues/175#event-6567229210

% guard S columns if latest feature ={guard} is not yet available
\newcommand*{\SiGuard}[1]{{#1}}

% standard environment
\SetTblrInner{
    hlines,
    vlines,
    columns={
            halign=c,
            valign=m,
        },
    measure=vbox,
}

% X columns
\NewTblrEnviron{tblrx}
\SetTblrInner[tblrx]{
    hlines,
    vlines,
    columns={
            halign=c,
            valign=m,
            co=1,  % coefficient of width for expendable columns (X columns)
        },
    width=\linewidth,
    vspan=minimal,
    measure=vbox,
}

% -X columns
\NewTblrEnviron{tblr-x}
\SetTblrInner[tblr-x]{
    hlines,
    vlines,
    columns={
            halign=c,
            valign=m,
            co=-1,  % shrinks X column down to natural width
        },
    width=\linewidth,
    vspan=minimal,
    measure=vbox,
}

% no hline and vline left and on top of first cell
\NewTblrEnviron{tblr_omit_first_cell}
\SetTblrInner[tblr_omit_first_cell]{
    hlines,
    vlines,
    columns={
            halign=c,
            valign=m,
        },
    hspan=even,
    vspan=minimal,
    %
    hline{1}={1}{white},  % first row, only first cell
    vline{1}={1}{white},
    measure=vbox,
}

% longtblr
\DefTblrTemplate{conthead-text}{default}{(Fortsetzung)}  % default: define and set at the same time
\DefTblrTemplate{contfoot-text}{default}{Fortsetzung auf nächster Seite}
\SetTblrStyle{caption-tag}{font=\bfseries}  % caption tag bold
\SetTblrInner[longtblr]{
    hlines,
    vlines,
    columns={
            halign=c,
            valign=m,
        },
    measure=vbox,
}


% biblatex
\addbibresource{files/JabRef_Database/physics.bib}

% manual header
\ihead{Versuchsname}  % inner (left) head
\chead{\textsc{Zach} Andreas (12004790)}  % center head
\ohead{Datum der Messung}  % outer (right) head



\begin{document}

\includepdf{input/deckblatt.pdf}

\clearpage
\tableofcontents
\newpage

\section{Aufgabenstellung}
\label{sec:aufgabenstellung}

\paragraph{Absatz}
\label{par:absatz}

Dies ist ein Testabsatz. Bitte diesen zu ignorieren.



\section{Grundlagen und Voraussetzungen}
\label{sec:grundlagen_voraussetzungen}

Text1\footcite[1000]{ref:dem1} \\
Text2\footcite[Kapitel 74]{ref:knoll} \\
Text3\footnote{\url{https://online.uni-graz.at/kfu_online/ee/ui/ca2/app/desktop/\#/login?$ctx=&redirect=Li4vLi4vLi4vZWUvdWkvY2EyL2FwcC9kZXNrdG9wLyMvc2xjLnRtLmNwL3N0dWRlbnQvY291cnNlcy82Mjg3OTk=}} \\  % must escape the # (\#), error otherwise!!
Text4\footcite{ref:genol2013}
%
% Gleichungsumgebung
\begin{equation}
    \label{eq:grenzen-oben-unten}
    \int_a^b x^2 \dd{x} =\frac{b^3-a^3}{3}  % \, -> small space, single character exponent doesn't need curly braces
\end{equation}
%
% inline math
Inline math: \(\lim_{n \to \infty} \left( 1 + \frac{1}{n} \right) ^{n} = e\)  \\ \\ % new, encouraged  % _ in inline math: no limits - in display math: limits
Inline math: $\lim \limits_{n \to \infty} \frac{1}{2n} = 0$  % old, discouraged
%
% display math
\begin{displaymath}
    \sqrt[3]{27} = 3 \implies \va{A} \cross \va{B} \implies \SI{30.0(2)}{\meter\per\second}
\end{displaymath}
%
Display math: \[\lim_{n \to \infty} \left( 1 + \frac{1}{n} \right) ^{n} = e\]  \\ \\ % new, encouraged  % _ in inline math: no limits - in display math: limits
\[ \int \frac{1}{x} \dd{x} = \ln|x| \qq{quad text, additionaly:} \dv{f}{x} \land \pdv{g}{y}\]  % new, encouraged - same as \begin{equation*} when using amsmath
$$ \vb*{F} = m \cdot \vb*{a} = \dot{\mup{p}}$$  % old, very discouraged

Test: \texttt{typewriter}

\subsection{Unsicherheitsberechnungen}
\label{subsec:unsicherheitsberechnungen}

Die explizit angegebenen Unsicherheiten der ermittelten Messgrößen basieren auf Berechnungen durch die Unsicherheitsangabe nach den Datenblättern der verwendeten Messgeräte. Diese sind in \autoref{tab:geraeteliste} vermerkt beziehungsweise referenziert.

Die Fehlerfortpflanzung der berechneten Werte basiert auf der Größtunsicherheitsmethode nach Gauß. Um diese Berechnungen zeiteffizient durchführen zu können, wird für jeden Unterpunkt der Laborübung ein Skript in \verb!Python! implementiert. Kernstück dessen ist das package \verb!uncertainties!\footcite{ref:uncertainties}, dass intern die Fehlerfortpflanzung berechnet. Gerundet wird nach den Angaben des Skriptums der Lehrveranstaltung \enquote{Einführung in die physikalischen Messmethoden}.\footcite{ref:messmethoden}



\section{Versuchsanordnung}
\label{sec:versuchsanordnung}



\section{Geräteliste}
\label{sec:geraeteliste}

\begin{table}[H]
    \centering
    \begin{samepage}  % caption and table on same page
        \caption[Geräteliste]{Verwendete Geräte und wichtige Materialien}  % optional argument for List of Tables, mandatory argument for caption
        \label{tab:geraeteliste}
        \begin{tblrx}{
                colspec={
                        *{3}{Q}  % multiplication syntax: *{n}{thing}
                        S[table-format=-2.2(2)]
                        Q[co=0]  % Q column for not overwriting standard values of preamble (no extendable width in order to work with diagbox)
                    },
                cell{2}{2}={r=2}{},  % empty specifier for not overwriting standard values of preamble
                cell{4}{3}={r=2,c=2}{},
                row{1}={guard},  % guard 1st row from being interpreted by a S column
            }
            Gerät   & Hersteller & Modell   & Unsicherheit  & Anmerkung                             \\
            Gerät 1 & ich        & meins    & 0.01(2)       & {quasi \\ perfekt \\ genau}           \\
            Gerät 2 &            & passt so & -21.4(13)     & quasi perfekt genau                   \\
            Gerät 3 & -          & passt so &               & $\nabla$                              \\
            Gerät 4 & -          &          &               & \diagbox[dir=NE]{Alle meine}{Entchen} \\  % if diagbox is not the widest column, width must be typeset manually -.-
        \end{tblrx}
    \end{samepage}
\end{table}



\section{Versuchsdurchführung und Messergebnisse}
\label{sec:versuchsdurchfuehrung_messergebnisse}

% example of longtblr
\begin{longtblr}[
        entry={Short entry},
        caption={Caption of example longtblr},
        label={tab:example_longtblr},
        note{a} = {note},  % footnote in table
    ]{}
    1 & 2 & 3 & 4 & 5 & 6 & 7.000.000.000.000.000.000.000.000\TblrNote{a} \\
    1 & 2 & 3 & 4 & 5 & 6 & 7.000.000.000.000.000.000.000.000             \\
    1 & 2 & 3 & 4 & 5 & 6 & 7.000.000.000.000.000.000.000.000             \\
    1 & 2 & 3 & 4 & 5 & 6 & 7.000.000.000.000.000.000.000.000             \\
    1 & 2 & 3 & 4 & 5 & 6 & 7.000.000.000.000.000.000.000.000             \\
    1 & 2 & 3 & 4 & 5 & 6 & 7.000.000.000.000.000.000.000.000             \\
    1 & 2 & 3 & 4 & 5 & 6 & 7.000.000.000.000.000.000.000.000             \\
    1 & 2 & 3 & 4 & 5 & 6 & 7.000.000.000.000.000.000.000.000             \\
    1 & 2 & 3 & 4 & 5 & 6 & 7.000.000.000.000.000.000.000.000             \\
    1 & 2 & 3 & 4 & 5 & 6 & 7.000.000.000.000.000.000.000.000             \\
    1 & 2 & 3 & 4 & 5 & 6 & 7.000.000.000.000.000.000.000.000             \\
    1 & 2 & 3 & 4 & 5 & 6 & 7.000.000.000.000.000.000.000.000             \\
    1 & 2 & 3 & 4 & 5 & 6 & 7.000.000.000.000.000.000.000.000             \\
    1 & 2 & 3 & 4 & 5 & 6 & 7.000.000.000.000.000.000.000.000             \\
    1 & 2 & 3 & 4 & 5 & 6 & 7.000.000.000.000.000.000.000.000             \\
\end{longtblr}


\section{Auswertung}
\label{sec:auswertung}



\section{Diskussion}
\label{sec:diskussion}



\section{Zusammenfassung}
\label{sec:zusammenfassung}



\addsec{Python-Skript}
\label{sec:python}

\lstinputlisting[language=Python]{input/python_example.py}



\clearpage
% Literaturverzeichnis
\printbibliography

% Abbildungsverzeichnis
\listoffigures

% Tabellenverzeichnis
\listoftables

\end{document}
