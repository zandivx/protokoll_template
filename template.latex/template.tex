% Protokoll-Template by Andreas Zach

\documentclass[11pt]{scrartcl}

% input and font
\usepackage[utf8]{inputenc}  % input in UTF-8
\usepackage[T1]{fontenc}  % output in T1 fonts (westeuropäische Codierung)
\usepackage{lmodern}  % latin modern fonts
\usepackage[ngerman]{babel}  % deutsches Sprachpaket, neue Rechtschreibung

% Seitengeometrie
\usepackage{scrlayer-scrpage}  % Seitenformatierung durch KOMA-interne Optionen
\usepackage[top=4cm, bottom=4cm]{geometry}  % Seitengeometrie

% Inhalt - Aussehen
\usepackage{amsmath, amssymb, amstext, mathtools}  % math packages (American Math Society) + Verbesserung von amsmath
\usepackage[hypcap=false]{caption, subcaption}  % caption editing - hypcap warning with hyperref
\usepackage{array}  % table editing
\usepackage[separate-uncertainty=true]{siunitx}  % bessere Darstellung von Einheiten - \pm soll \pm bleiben und nicht ..(.)
\usepackage{icomma}  % Kommata in Dezimalzahlen verursachen keinen Abstand mehr

% einfügen
\usepackage{graphicx}  % Bilder einfügen
\usepackage{pdfpages}  % PDF als vollständige Seiten einfügen
\usepackage{lastpage}  % referenziert die letzte Seite

% Literaturverwaltung
\usepackage[style=apa]{biblatex}  % backend=biber is standard
\usepackage{csquotes}  % better quotation - should also be used in combination with package babel (warning)
\usepackage{xurl}  % breaks links - after biblatex, but before hyperref!
\usepackage[hidelinks]{hyperref}  % angeblich errorlastig

% KOMA setups
% Kopf- und Fußzeile durch KOMA
\pagestyle{scrheadings}  % KOMA darf entscheiden
\clearpairofpagestyles  % reset
\setkomafont{pageheadfoot}{\normalfont}  % Standardschrift in Kopf- und Fußzeile
\setlength{\headheight}{27.2pt}  % just look at the warning
\ihead{Versuchsname}  % Kopf links (inner)
\chead{\textsc{Zach} Andreas \\ 12004790}  % Kopf mittig (center)
\ohead{Datum der Messung}  % Kopf rechts (outer)
\cfoot{\pagemark{} / \pageref{LastPage}}  % Fuß Mitte
% {}: empty statement
% \ : protected space
% \,: small space
\DeclareTOCStyleEntry{dottedtocline}{section}  % Inhaltsverzeichnis mit Punkten (nur sections)

% other packages
% array
\renewcommand{\arraystretch}{1.2}  % stretched die Zellengröße in tables auf das 1,2-fache
% siunitx
\sisetup{locale = DE}
% biblatex source
\addbibresource{C:/Users/andre/Dropbox/Uni/Physik/Allgemeines/protokoll_template/files/JabRef_Database/physics.bib}  % database import with absolute path - .bib!

% neue Namen (babel überschreibt Standardbezeichnungen aus dem Englischen in der Präambel - therefore gehören die neuen commands erst am Dokumentbeginn ausgeführt)
\AtBeginDocument{\renewcommand{\refname}{Literaturverzeichnis}}
% weitere:
%\contentsname
%\listtablename
%\listfigurename

% eigene Commands
%\newcommand{\der}[3][]{\frac{\mathrm{d}^{#1}#2}{\mathrm{d}#3^{#1}}}  % derivative
\newcommand{\pder}[3][]{\frac{\partial^{#1}#2}{\partial#3^{#1}}}  % partial derivative



\begin{document}

%\includepdf{deckblatt.pdf}

\clearpage
\tableofcontents
\newpage

\section{Aufgabenstellung}
\label{sec:aufgabenstellung}

\section{Grundlagen und Voraussetzungen}
\label{sec:grundlagen_voraussetzungen}

Text1\footcite[1000]{ref:dem1} \\
Text2\footcite[Kapitel 74]{ref:knoll} \\
Text3\footnote{\url{https://online.uni-graz.at/kfu_online/ee/ui/ca2/app/desktop/\#/login?$ctx=&redirect=Li4vLi4vLi4vZWUvdWkvY2EyL2FwcC9kZXNrdG9wLyMvc2xjLnRtLmNwL3N0dWRlbnQvY291cnNlcy82Mjg3OTk=}} \\  % must escape the # (\#), error otherwise!!
Text4\footcite{ref:genol2013}


\section{Versuchsanordnung}
\label{sec:versuchsanordnung}

% Gleichungsumgebung
\begin{equation}
\label{eq:grenzen-oben-unten}
    \int \limits_{a}^{b} x^2 \, dx =\frac{b^3-a^3}{3}  % \, liefert quasi ein Leerzeichen - single character exponent muss nicht in {}-Klammern stehen
\end{equation}

% inline math
\(\lim \limits_{n \to \infty} \left( 1 + \frac{1}{n} \right) ^{n} = e\)  \\ % new, encouraged
$\lim \limits_{n \to \infty} \frac{1}{2n} = 0$  % old, disencouraged

% display math
\begin{displaymath}
    \frac{df}{dx} = f'
\end{displaymath}

\[ \int \frac{1}{x} \, dx = \ln|x| \]  % new, encouraged - same as \begin{equation*} when using amsmath
$$ \textbf{F} = m \cdot \textbf{a} $$  % old, very disencouraged

\section{Geräteliste}
\label{sec:geraeteliste}

% Beispieltabelle 
\begin{center}
\captionof{table}[Geräteliste]{Verwendete Geräte + genaue, lange Beschreibung}  % optionales Argument wird in Verzeichnissen verwendet, verpflichtendes Argument direkt im Text
\label{tab:geraeteliste}
    \begin{tabular}{|c|c|c|c|} \hline
        Gerät & Hersteller & Modell & Genauigkeitsklasse \\ \hline
        Der Gerät & Schweißfrei Inc. & Nie-Müde & 12 \\ \hline
    \end{tabular}
\end{center}

\section{Versuchsdurchführung und Messergebnisse}
\label{sec:versuchsdurchfuehrung_messergebnisse}

\section{Auswertung}
\label{sec:auswertung}

\section{Diskussion und Zusammenfassung}
\label{sec:diskussion_zusammenfassung}

% Literaturtabelle
\printbibliography

% Abbiludngsverzeichnis
\listoffigures

% Tabellenverzeichnis
\listoftables

\end{document}
